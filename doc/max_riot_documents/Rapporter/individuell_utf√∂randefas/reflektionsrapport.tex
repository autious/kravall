\documentclass[a4paper]{report}

\usepackage[utf8]{inputenc}
\usepackage[swedish]{babel}
%\usepackage{hyperref}
\usepackage{graphicx}
%\usepackage{csquotes}

\title{Individuell reflektionsrapport: utförandefas, grupp 5}
\author{Max Danielsson}
\date{17 januari 2014}
\begin{document}
   \maketitle

\section*{Introduktion}

Detta dokument är en rapport konstruerad runt fyra huvudsakliga frågor gällande pågående projekt:

\begin{quote}
    \begin{itemize}
        \item Hur följer ni er metod idag, vad har ni med och vad har ni problem med att följa ?
        \item På vilket sätt kan ni bevisa att ni är en effektiv grupp?
        \item Kommer ni att lyckas nå målen som ni satta för ert projekt?
        \item Hur kan ni bevisa att ert projekt har den kvalitet som ni sa att ni skulle ha i början av ert projekt?
    \end{itemize}
\end{quote}

Frågorna bearbetas efter listad ordning med en sektion per fråga.

\section*{Metod}

Vår metod som sattes upp tidigt under projektets gång har inte modifierats
något märkvärt. Vi jobbar fortfarande med en väldigt fysisk variant av scrum
där allt arbete planeras upp och följs upp med hjälp av en scrumboard och en
burndown chart. Vi har i snitt hittils lagt ned ungefär en dag per sprint att
planera upp sprinten vilken tar en märkbar mängd med tid, speciellt med tanke
på att alla inom arbetsgruppen för med mesta är närvarande och deltagande i
planeringen. Ett alternativ som effektiviserar och möjligör mindre deltagande
med samma resultat har hittils inte föreslagits eller utvecklats.

Under sprintets gång så jobbar folk till stor del individuellt med den
planering som satts upp för att göra klart alla scrumboard tasks. Många
aspekter av projektet kräver dock ofta frekvent sammarbete och där har den
obligatoriska närvaron (kl. 9-17 mån-fre) hjälpt med att miniera den tid som
krävs för att genomför dessa kollaborationer.

På vår scrumboard så har vi ett "kanban" tillägg, en pipeline med flera steg
där var steg skall ge utvecklaren en kontext för vilka aspekter som bör
berarbetas i ordning. Perosonligen så har jag på programmeringssidan märkt att
denna pipeline inte har följts. Detta har mestadels påverkat utvecklingen av
unit test som har stagnerat under de senaste 3-4 veckorna, men även delvis
dokumentation och kommentering av kodbasen. Denna avvikning från satt standard
beror troligen på att vi utvecklare lätt fallit in i vara gamla arbetsvanor så
fort någonting skall göras och det inte finns något större utrymme för att
misslyckas eller fördröja utvecklingen, för mycket har planerat med för lite
tid, vilket märks i att vi inte fullständigt uppnått ett enda sprintmål hittils
. Detta har fungerat väl och vi har hittils varit effektiva i att leverera
funktion. Men oron är att kvalitén på produkten och arbetet kan lida i längden,
om man extrapolerar ut utanför detta projektets ramar. Detta innebär att för
vad vi försöker tillverka, inom de ramar vi har givits så tror jag att den
faktiska metoden som har resulterat har givit mer värde i produkten än vad som
annars skulle uppnåtts, men är inte lämplig för ett större projekt.

\section*{Effektivitet}

Vi har under var sprint dagligen mätt upp hur mycket arbete som genomförts och
har jämfört denna siffran med vad som har planerats. Detta ger oss en velocity
som underlättar när vi vidare skall planera in nästa sprints arbetsbörda. Med
hjälp av detta mått så kan vi även få lite mer koll på eventuella avvikelser i
vår arbetsprestation. Vi kan även se om arbetsgruppen erhar en ökad
arbetsförmåga allteftersom tiden skrider frammåt och mängden arbete som utförs
ökar.

Vidare, att kunna bevisa eller visa upp någon form av effektivitet är ett
problematisk uppdrag på grund av dess relativa betydelse i en arbetsform som är
ytters ung och konstant förändrande. Sedan så kan frågan besvaras på minst två
nivåer, primärt hur vi presterar inom vår personliga förmåga och för det andra
hur vi presterar jämfört ett globalt markandsideal. Mest logiskt är dock att
jämföra oss med liknande projekt som genomförts inom samma kurs.

\section*{Mål}

Tidigt i vecka tre så spenderades två dagar på att titta frammåt i tiden och
planerat upp de kommande sex veckorna för att kunna få en bättre känsla för
hurvida vi kan uppnå de mål vi har med projektet. Så som planeringen nu är
upplagd så satsar vi på att uppnå ett feature komplett spel i slutet av
feburari vilket ger oss 3 veckors marginal att titta på den produkt som
utvecklats för att framhäva och polera de aspekter som tros kunna ge mest
värde till spelet.

Vi känner oss positiva när det kommer till att uppnå målet, men det finns inte
mycket marginal för större förhinder. Det finns en risk att vissa aspekter,
speciellt AI'n i projektet, kan komma att ta längre tid än väntat vilket direkt
påverkar spelets grundläggande funktion och kvalitét. Det kan bli så att vi är
tvugna att ta bort vissa aspekter av spelet. Med hjälp av den nya långående
planeringen så kommer behovet av dess förändringar förhoppningsvis göra sig
tydliga tidigare än vad de annars skulle varit, vilket ger oss en möjlighet att
rädda spelet som helhet genom att justera slutmålet. Vår "feature ladder"
hjälper även till att prioritera features som kan tas bort innan andra utan
att det får det spelmekaniska att kollapsa.

\section*{Visa uppnåd förväntad kvalitét}

Tydliga kvalitativa mål sattes aldrig upp i början av projektet. Detta till
stor del för att det inte finns något tidigare arbete utfört av den sammansatta
gruppen som man eventuellt kan använda som måttstock vilket försvårar
processen. Den kvalitét som istället förväntas kommer från individers åsikter,
synpunkter och förmåga igenom projektets gång. Jag hoppas att spelets
underliggande kvalitét är tillräckligt stor så att spelet kan utvärderas
primärt på sina spelmekaniska aspekter istället för eventuella tekniska
brister.

För ett användbart mått av kvalité i spelet så måste spelet speltestas av en
målgrupp som utvärderar spelet subjektivt så vi kan få en marknadsundersökning
på potentiell försäljning, vilket i grund och botten är den enda kvalitativa
aspekten som är relevant från ett marknadsperspektiv.
\end{document}
